\documentclass[12pt,a4paper]{article}
\usepackage[italian, english]{babel}
\usepackage[utf8]{inputenc}
\usepackage[T1]{fontenc}
\usepackage{amsmath}
\usepackage{amsfonts}
\usepackage{xurl}
\usepackage{graphicx}
\usepackage[dvipsnames]{xcolor}
\usepackage{tikz}\usepackage{multirow}
\usepackage{comment}
\usepackage{tabularx}
\usepackage{multirow}

\makeatletter
\newcommand*{\rom}[1]{\expandafter\@slowromancap\romannumeral #1@}
\makeatother

%=====================INIZIO DOCUMENTO=====================
\begin{document}

%%%%%%%%%%%%%%%% header %%%%%%%%%%%%%%%%%%%%%%%%%%%%%%

\noindent\begin{minipage}{0.3\textwidth}
    \includegraphics[width=\linewidth]{logo.png}
\end{minipage}%
\hfill%
\begin{minipage}{0.6\textwidth}\raggedright
    \huge
    ERROR\_418\\
    Verbale 26/10/23
\end{minipage}

%%%%%%%%%%%%%%%% Sezione informativa %%%%%%%%%%%%%%%%%%
\large
\setlength{\extrarowheight}{9pt}
\raggedright
\begin{tabularx}{0.9\textwidth} [right] {
        >{\raggedright\arraybackslash}X
        >{\raggedright\arraybackslash}X
    }
    Mail:           & error418swe@gmail.com                              \\
    Redattori:      & Antonio Oseliero, Alessio Banzato                  \\
    Verificatori:   & Riccardo Carraro, Giovanni Gardin, Rosario Zaccone \\
    Amministratori: & Silvio Nardo, Mattia Todesco                       \\
    Referente aziendale: & Matteo Bassani (Sanmarco Informatica) \\
    Destinatari:    & T. Vardanega, R. Cardin
\end{tabularx}
%%%%%%%%%%%%%%%%%%%%%% Presenze %%%%%%%%%%%%%%%%%%%%%%%%
\vspace{3mm}
\hline
\raggedright
\begin{tabular}{c c}
    \multicolumn{2}{c}{Inizio Meeting: 18:00 \hspace{4mm}
    Fine Meeting: 18:20 \hspace{4mm} Durata: 20min} \\
    Presenze: &                                    \\
\end{tabular}

\begin{center}
    \begin{tabular}{ |c|c|c|c|c| }
        \hline
        Nome     & Durata Presenza &  & Nome     & Durata Presenza \\
        \hline
        Antonio  & 20min       &  & Alessio  & 20min           \\
        \hline
        Riccardo & 20min       &  & Giovanni & 20min           \\
        \hline
        Rosario  & /           &  & Silvio   & 20min           \\
        \hline
        Mattia   & 20min       &  & &                 \\
        \hline
    \end{tabular}

\end{center}

\newpage

\section{Ordine del giorno}

\begin{itemize}
    \item primo contatto con Matteo Bassani, rappresentante dell'azienda proponente Sanmarco Informatica e nostro referente per il capitolato C5;
    \item richiesta di precisazioni tecniche sui requisiti del capitolato.
\end{itemize}

\section{Chiarimenti sul capitolato C5}
Dopo esserci presentati abbiamo chiesto maggiori informazioni all'azienda Sanmarco Informatica riguardo ai nostri dubbi sul capitolato C5: Warehouse management 3D (WMS3).
Di seguito sono riportati i temi trattati.

\paragraph{Notifiche}
Abbiamo chiarito che, quando si parla di notifiche di trasferimento, si intende l'invio di una notifica tramite REST API.

\paragraph{Scaffali e bin}
Gli scaffali sono divisi in ripiani e i ripiani sono a loro volta suddivisi in bin. In uno stesso scaffale possono coesistere bin di dimensioni diverse (su ripiani diversi). I bin dello stesso ripiano hanno dimensioni uguali. I bin sono aree generiche dove gli articoli possono essere posizionati (sul pavimento, sui veicoli, sugli scaffali, ...); tuttavia, per soddisfare le richieste minime è sufficiente lavorare con bin posizionati su scaffali.

\paragraph{Controlli di sicurezza}
Controllare se un articolo sia compatibile con un bin (per posizione o per articoli limitrofi) o meno è una feature interessante ma non richiesta dal capitolato.

\paragraph{Persistenza dei dati}
Il programma dovrà precaricare una disposizione di scaffali e articoli da un database relazionale. Non è necessario che il programma offra la persistenza dei dati: ad ogni nuova esecuzione, il programma riparte dallo stato iniziale predefinito. I DBMS consigliati per la gestione del database sono di tipo relazionale: PostgreSQL, MariaDB, MySQL e SQL Server.

\paragraph{Movimentazione scaffali nello spazio 3D} La possibilità di creare, modificare e spostare gli scaffali manualmente nello spazio 3D è una feature utile che va oltre le richieste minime. Si tratta di funzionalità supplementari, poiché non è richiesto che lo stato del magazzino venga salvato in memoria.

\section{Altri argomenti affrontati}

\paragraph{Frequenza dei meeting}
Abbiamo chiarito quale possa essere la frequenza dei meeting tra fornitore e proponente. È emerso che i meeting dovranno avere almeno cadenza mensile, ma sarà certamente possibile chiedere colloqui più frequenti. Il referente si è inoltre reso disponibile per rispondere a qualsiasi esigenza tramite posta elettronica.

\paragraph{Pareri ed opinioni}
La riunione si è conclusa con una veloce discussione dei motivi che ci hanno spinto a valutare il capitolato C5 come prima scelta. In maniera unanime, abbiamo concordato che lavorare su un progetto 3D è molto stimolante. Inoltre, si tratta di un'interessante evoluzione dei software logistici attuali, e per questo crediamo nella sua utilità.

\newpage

\begin{tabular}{@{}p{.5in}p{4in}@{}}
\small{Approvato:} & \hspace{0.3cm} \hrulefill \\
&  \hspace{0.3cm} \small{Matteo Bassani, Sanmarco Informatica}\\
\end{tabular}

\end{document}
